% Created 2020-09-16 Wed 18:20
% Intended LaTeX compiler: pdflatex
\documentclass[10pt, compress, aspectratio=169, xcolor={table,usenames,dvipsnames}]{beamer}

\usepackage{booktabs}
\mode<beamer>{\usetheme[numbering=fraction, progressbar=none, titleformat=smallcaps, sectionpage=none]{metropolis}}
\usepackage{sourcecodepro}
\usepackage{booktabs}
\usepackage{array}
\usepackage{listings}
\usepackage{multirow}
\usepackage{caption}
\usepackage{xeCJK}
\usepackage{graphicx}
\usepackage[english]{babel}
\usepackage[scale=2]{ccicons}
\usepackage{hyperref}
\usepackage{relsize}
\usepackage{amsmath}
\usepackage{bm}
\usepackage{wasysym}
\usepackage{ragged2e}
\usepackage{textcomp}
\usepackage{pgfplots}
\usepackage{appendixnumberbeamer}
\usepgfplotslibrary{dateplot}
\definecolor{Base}{HTML}{191F26}
\definecolor{Adversarial}{HTML}{FF8D6D}
\definecolor{Benign}{HTML}{01A982}
\definecolor{Highlight}{HTML}{ffda99}
\definecolor{Accent}{HTML}{bb0300}
\setbeamercolor{alerted text}{fg=Accent}
\setbeamercolor{frametitle}{fg=Base,bg=normal text.bg}
\setbeamercolor{normal text}{bg=black!2,fg=Base}
\setsansfont[BoldFont={Source Sans Pro Semibold},Numbers={OldStyle}]{Source Sans Pro}
\lstdefinelanguage{Julia}%
{morekeywords={abstract,struct,break,case,catch,const,continue,do,else,elseif,%
end,export,false,for,function,immutable,mutable,using,import,importall,if,in,%
macro,module,quote,return,switch,true,try,catch,type,typealias,%
while,<:,+,-,::,/},%
sensitive=true,%
alsoother={$},%
morecomment=[l]\#,%
morecomment=[n]{\#=}{=\#},%
morestring=[s]{"}{"},%
morestring=[m]{'}{'},%
}[keywords,comments,strings]%
\lstdefinelanguage{dockerfile}{
keywords={FROM, RUN, COPY, ADD, ENTRYPOINT, CMD,  ENV, ARG, WORKDIR, EXPOSE, LABEL, USER, VOLUME, STOPSIGNAL, ONBUILD, MAINTAINER},
sensitive=false,
comment=[l]{\#},
morestring=[b]',
morestring=[b]"
}
\lstdefinelanguage{yaml}{
keywords={true,false,null,y,n},
ndkeywords={},
sensitive=false,
comment=[l]{\#},
morecomment=[s]{/*}{*/},
morestring=[b]',
morestring=[b]"
}
\lstset{ %
backgroundcolor={},
basicstyle=\ttfamily\scriptsize,
breakatwhitespace=true,
breaklines=true,
captionpos=n,
commentstyle=\color{Accent},
extendedchars=true,
frame=n,
keywordstyle=\color{Accent},
rulecolor=\color{black},
showspaces=false,
showstringspaces=false,
showtabs=false,
stepnumber=2,
stringstyle=\color{gray},
tabsize=2,
}
\renewcommand*{\UrlFont}{\ttfamily\smaller[2]\relax}
\graphicspath{{../../img/}}
\addtobeamertemplate{block begin}{}{\justifying}
\captionsetup[figure]{labelformat=empty}
\usetheme{default}
\author{ \footnotesize Pedro Bruel, Giuliano Belinassi}
\date{ \scriptsize 16 de Set, 2020}
\title{Exemplos em Assembly}
\hypersetup{
 pdfauthor={ \footnotesize Pedro Bruel, Giuliano Belinassi},
 pdftitle={Exemplos em Assembly},
 pdfkeywords={},
 pdfsubject={},
 pdfcreator={Emacs 27.1 (Org mode 9.2.5)},
 pdflang={English}}
\begin{document}

\maketitle

\section{Introdução}
\label{sec:org055d33d}
\begin{frame}[label={sec:orgef38c21}]{Roteiro}
\begin{block}{Objetivo}
\begin{itemize}
\item Desenvolver mais familiaridade com o processo de compilação e com seus produtos
\item Para isso, veremos  exemplos com linguagens de alto-nível,  Makefiles, e com o
\emph{shell} do Linux
\end{itemize}
\end{block}
\begin{block}{Tópicos dos Exemplos}
\begin{enumerate}
\item Gerando assembly com Julia
\item Gerando assembly com GCC e Clang
\item Observando otimizações feitas pelo compilador
\end{enumerate}
\end{block}
\begin{block}{Repositório no GitHub}
\begin{itemize}
\item Slides e código
\item Baixe o repositório para acompanhar a aula!
\item \url{https://github.com/phrb/presentations/tree/master/org/assembly}
\end{itemize}
\end{block}
\end{frame}
\begin{frame}[label={sec:org7f16386},fragile]{Recursos Adicionais}
 \begin{block}{Grupo na USP: \href{https://flusp.ime.usp.br/}{FLUSP (FLOSS at USP)}}
\end{block}
\begin{block}{Curso no Youtube: \href{https://www.youtube.com/playlist?list=PLUl4u3cNGP63VIBQVWguXxZZi0566y7Wf}{MIT Performance Engineering for Software Systems}}
\end{block}
\begin{block}{Assembly \texttt{x64}}
\begin{itemize}
\item Cheat Sheets: \href{https://cs.brown.edu/courses/cs033/docs/guides/x64\_cheatsheet.pdf}{x64}, \href{https://devhints.io/makefile}{Makefile}
\item \href{https://www.wiley.com/en-us/Assembly+Language+Step+by+Step\%3A+Programming+with+Linux\%2C+3rd+Edition+-p-9781118080993}{Assembly Language Step-by-Step}, Duntemann
\item \href{https://www.intel.com/content/dam/www/public/us/en/documents/manuals/64-ia-32-architectures-software-developer-instruction-set-reference-manual-325383.pdf}{Manual Intel}
\end{itemize}
\end{block}
\begin{block}{Livros}
\begin{itemize}
\item \href{https://www.charlespetzold.com/code/}{Code}, Petzold
\item \href{https://nostarch.com/writegreatcode1\_2e}{Write Great Code}, Hyde
\item \href{https://www.r-5.org/files/books/computers/compilers/writing/Keith\_Cooper\_Linda\_Torczon-Engineering\_a\_Compiler-EN.pdf}{Engineering a Compiler}, Cooper \& Torczon
\item \href{https://acs.pub.ro/\~cpop/SMPA/Computer\%20Architecture\%20A\%20Quantitative\%20Approach\%20(5th\%20edition).pdf}{Computer Architecture}, Hennessy \& Patterson
\item \href{https://www.pearson.com/us/higher-education/program/Aho-Compilers-Principles-Techniques-and-Tools-2nd-Edition/PGM167067.html}{Compilers}, Aho, Lam, Sethi, Ullman
\end{itemize}
\end{block}
\end{frame}
\begin{frame}[label={sec:org23951a7},fragile]{Gerando Assembly com Julia}
 \begin{block}{A linguagem Julia}
\begin{itemize}
\item Compilação \alert{Just-in-Time} usando o LLVM
\item Podemos olhar o código que será executado com a função \texttt{code\_native}
\end{itemize}
\end{block}
\begin{block}{Exemplos}
\begin{enumerate}
\item Gerando código para \alert{tipos diferentes}: \texttt{examples/julia/simple.jl}
\item \alert{``Mágicas''} do compilador? \texttt{examples/julia/arithmetic.jl}
\end{enumerate}
\end{block}
\end{frame}
\begin{frame}[label={sec:org5752255},fragile]{Gerando Assembly com GCC e Clang}
 \begin{block}{Compilação Interativa}
\begin{enumerate}
\item \href{https://godbolt.org/}{Compiler Explorer}
\item Usando um script: \texttt{examples/interactive\_compilation/watch\_asm.sh}
\item Pode funcionar pra qualquer linguagem
\end{enumerate}
\end{block}
\end{frame}
\begin{frame}[label={sec:orga68d387},fragile]{Observando Otimizações do Compilador}
 \begin{block}{Exemplos fornecidos pelo \alert{Giuliano}!}
\begin{enumerate}
\item (Des)complicando o GCC: \texttt{examples/gcc\_subprograms/LEIAME.txt}
\item \(sin(atan(x))\): \texttt{examples/sinatan\_optimize/LEIAME.txt}
\item Soma de PA: \texttt{examples/arithmetic\_progression/LEIAME.txt}
\end{enumerate}
\end{block}
\end{frame}
\maketitle
\end{document}
